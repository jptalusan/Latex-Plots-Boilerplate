% load the data table ...
\pgfplotstableread[col sep=comma]{data/bar_data_small.csv}{\loadedtable}
    % and store the number of columns in `\NoOfCols'
    % (minus 1 because counting in `\foreach' starts with zero
    \pgfplotstablegetcolsof{\loadedtable}
    \pgfmathtruncatemacro{\NoOfCols}{\pgfplotsretval-1}

\begin{figure}
\centering
\begin{tikzpicture}
    \begin{axis}[
        % adjust the `width' a bit by keeping the default `height'
        % width=1.2*\axisdefaultwidth,
        % height=\axisdefaultheight,
        width=1.0\linewidth,
        % set appropriate `ymax' value so the `nodes near coords' fit to the plot
        % (adjusting the `ymin' value is just to make it look a little bit better)
        % ymin=1e-1,
        % ymax=1e6,
        % there should be no gap between the bars in one group
        ybar=0pt,
        % use data from the table for the xticklabels
        xtick=data,
        xticklabels from table={\loadedtable}{COUNT},
        % to start the bars from the bottom edge of the plot
        % (otherwise they would start from 10^0
        %  borrowed from <http://tex.stackexchange.com/a/86688/95441)
        % log origin=infty,
        % adjust the size of the bars so they don't overlap
        % (you can play with the numerator to change the gap between the groups)
        % bar width=0.85/\NoOfCols,
        % enlarge the x limits so all of the bars are shown
        % (play with the value to adjust the gap on the sides of the plot)
        enlarge x limits={abs=0.6},
        % and position the legend outside of the plot to not overlap with the data
        % legend pos=outer north east,
        legend style={at={(0.03,0.9)},anchor=west},
        % add `nodes near coords'
        % nodes near coords={
        %     % because internally PGFPlots works with floating point numbers, we
        %     % change them to fixed point numbers
        %     \pgfkeys{
        %         /pgf/fpu=true,
        %         /pgf/fpu/output format=fixed,
        %     }%
        %     % check if numbers are greater than 1000 and if so, divide them by
        %     % 1000 to convert them from ms to s scale
        %     \pgfmathparse{
        %         ifthenelse(
        %             \pgfplotspointmeta < 1000,
        %             \pgfplotspointmeta,
        %             \pgfplotspointmeta/1000
        %         )
        %     }%
        %     % to now decide which of the two cases we have, we compare the
        %     % point meta value, but because `\ifnum' compares integers, we first
        %     % have to convert the fixed number to an integer
        %         \pgfmathtruncatemacro{\Y}{\pgfplotspointmeta}%
        %     \ifnum\Y<1000
        %         \pgfmathprintnumber{\pgfmathresult}\,ms
        %     \else
        %         \pgfmathprintnumber{\pgfmathresult}\,s
        %     \fi
        % },
        % set the style of the `nodes near coords'
        % nodes near coords style={
        %     font=\tiny,
        %     rotate=90,
        %     anchor=west,
        % },
        % as basis for the `nodes near coords' use the raw y values
        % point meta=rawy,
        % Distances between ticks
        xtick distance=2,
    ]
        % add the data rows
        \foreach \i in {1,...,\NoOfCols} {
            \addplot table [
                x expr=\coordindex,
                y index=\i,
                col sep=comma,
            ] {\loadedtable};
                % to automatically add the legend entries first extract the
                % column name and store it in `\colname'
                % (this is an undocumented command so far. I borrowed it from
                %  <http://tex.stackexchange.com/q/171021/95441>)
                    \pgfplotstablegetcolumnnamebyindex{\i}\of{\loadedtable}\to{\colname}
                % now you can add the legend entry
                % (because we are in a loop we have to use the "expanded" version)
                \addlegendentryexpanded{\colname};
        }
    \end{axis}
\end{tikzpicture}
\caption{Bar Plot From CSV}
\end{figure}