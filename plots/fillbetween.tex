\pgfplotstableread[col sep=comma]{data/fig_1_to_jp_r_3.csv}{\routethree}

\begin{figure}[htbp]
\centering
\begin{tikzpicture}
\begin{axis}[
    xlabel={Trip Load},
    ylabel={Number of\\Trips},
    % xmin=1, xmax=31, % Adjust the x-axis range to fit the number of days in the month
    ymin=0, % Adjust the y-axis range as needed
    grid=both,
    grid style={line width=.1pt, draw=gray!10},
    major grid style={line width=.2pt,draw=gray!50},
    % minor tick num=4,
    enlarge x limits=false,
    enlarge y limits={upper,value=0.1},
    legend pos=north east,
    width=0.45\textwidth,
    height=2in,
    % Adjust steps
    % xtick distance=2
    % Changing label format 1:
    % /pgf/number format/.cd,
    %         use comma,
    %         1000 sep={}
    % Changing label format 2:
    scaled y ticks=base 10:-3,
    ytick scale label code/.code={},
    yticklabel={\pgfmathprintnumber{\tick}k},
    % NEeded to new line the label
    ylabel style={align=center}
]

% Create a named path for the horizontal axis.
% https://www.sqlpac.com/en/documents/latex-pgfplots-tikz-filling-areas-under-and-between-curves.html
\path [name path=xaxis]
      (\pgfkeysvalueof{/pgfplots/xmin},0) --
      (\pgfkeysvalueof{/pgfplots/xmax},0);
      
% Plot the line
\addplot[blue, name path=a, no markers, smooth, thick] table[x=trip_load, y=transit_date, col sep=comma]{\routethree};
\legend{Route 3}

\addplot+[blue!30, opacity=0.4] fill between [of=a and xaxis, soft clip={domain=-1:10}];
% \legend{Value2}


\end{axis}
\end{tikzpicture}
\caption{Adding Fill between}
\label{fig:heavy_tail}
\end{figure}