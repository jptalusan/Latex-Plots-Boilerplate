% https://tex.stackexchange.com/questions/102385/how-to-draw-a-return-arrow-from-node-3-to-node-1
% https://tex.stackexchange.com/questions/539468/how-to-define-a-maximum-height-for-a-tikz-decision-node
% https://www.overleaf.com/learn/latex/LaTeX_Graphics_using_TikZ%3A_A_Tutorial_for_Beginners_(Part_3)%E2%80%94Creating_Flowcharts
\newcommand\nodebg{white!100}

\tikzstyle{startstop} = [rectangle, rounded corners, minimum width=3cm, minimum height=1cm,text centered, draw=black, fill=\nodebg, text width=3cm]

\tikzstyle{io} = [trapezium, trapezium left angle=70, trapezium right angle=110, minimum width=3cm, minimum height=1cm, text centered, draw=black, fill=\nodebg]

\tikzstyle{process} = [rectangle, minimum width=3cm, minimum height=1cm, text centered, text width=3cm, draw=black, fill=\nodebg]

\tikzstyle{decision} = [diamond, minimum width=3cm, minimum height=1cm, text centered, draw=black, fill=\nodebg, text width=2cm, aspect=1.4]

\tikzstyle{arrow} = [thick,->,>=stealth]

\tikzset{
  block/.style={rectangle,draw,minimum size=1cm},
  line/.style={draw, -latex},
}

\begin{figure}
% \centering
\begin{tikzpicture}[node distance=0.5cm, 
  scale=0.8, transform shape,
  % auto,
  /utils/temp/.style={#1/.style={to path={r-#1(\tikztotarget)\tikztonodes}}},
  /utils/temp/.list={ud, rl, du, lr}
  ]
\node (start) [startstop] {\texttt{execute()}};
\node (dec1) [decision, below=of start] {EventQueue Empty?};

\node (pro1) [process, right=of dec1] {Simulation finished};
\node (pro2) [process, below=of dec1] {Execute all events in current timestep};
\node (pro3) [process, below=of pro2] {Update environment and state};

\node (dec2) [decision, below=of pro3] {Did cars arrive?};
\node (pro4) [process, right=of dec2] {Assign cars to chargers};
\node (pro5) [process, below=of dec2] {Get charger actions from policy};
\node (pro6) [process, below=of pro5] {Take policy action and update state};

\node (pro7) [process, below=of pro6] {Log state and actions};
\node (pro8) [process, below=of pro7] {Increment timestep};


\draw [arrow] (start) -- (dec1);
\draw [arrow] (dec1) -- node[anchor=south] {yes} (pro1);
\draw [arrow] (dec1) -- node[anchor=east] {no} (pro2);
\draw [arrow] (pro2) -- (pro3);
\draw [arrow] (pro3) -- (dec2);
\draw [arrow] (dec2) -- node[anchor=south] {yes} (pro4);
\draw [arrow] (dec2) -- node[anchor=east] {no} (pro5);
\draw [arrow] (pro4) |- (pro5);

\draw [arrow] (pro5) -- (pro6);
\draw [arrow] (pro6) -- (pro7);
\draw [arrow] (pro7) -- (pro8);
\path [line] (pro8) edge[lr] (start);

% \draw [arrow] (pro8) -- (start);


% \draw [arrow] (start) -- (in1);
% \draw [arrow] (in1) -- (pro1);
% \draw [arrow] (pro1) -- (dec1);
% \draw [arrow] (dec1) -- (pro2a);
% \draw [arrow] (dec1) -- (pro2b);

% \draw [arrow] (dec1) -- node[anchor=east] {yes} (pro2a);
% \draw [arrow] (dec1) -- node[anchor=south] {no} (pro2b);

% \draw [arrow] (pro2a) -- (out1);
% \draw [arrow] (out1) -- (stop);
% \draw [arrow] (pro2b) |- (pro1);

\end{tikzpicture}
\caption{Chart drawn using TikZ}
\end{figure}