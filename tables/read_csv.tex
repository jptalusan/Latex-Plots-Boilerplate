
\begin{filecontents}{testdata.csv}
theta, x, y
, jp, test
30.0,   0.0,    0.0
60.0,   1.9098, 5.8779
90.0,   6.9098, 9.5106
120.0,  13.09,  9.5106
150.0,  18.09,  5.8779
180.0,  20.0,   0.0
\end{filecontents}

\pgfplotstableread[col sep=comma]{testdata.csv}{\table}

\begin{table*}
\centering
\caption{Reading Table from a File with \textbf{siunitx}.}
\pgfplotstabletypeset[
    % dec sep align=S,    % Use the siunitx `S` column type for aligning at decimal point
    fixed,fixed zerofill,     % Fill numbers with zeros
    precision=3,        % Set number of decimals
    display columns/0/.style={
        precision=1,    % Change for first column (column index 0)
        column name=$\theta_{2,i}$
    },
    % display columns/1/.style={column name=$X$,column type={S}},
    % display columns/2/.style={column name=$Y$,column type={S}},
    every row 0 column 1/.style={string type},
    every row 0 column 2/.style={string type},
    every head row/.style={before row=\toprule, after row=\midrule},
    every last row/.style={after row=\bottomrule},
  every row 2 column 1/.style={
    postproc cell content/.style={
      @cell content/.add={$\bf}{$}
    }
  },
    ] {\table}
\label{table:read_csv}
\end{table*}